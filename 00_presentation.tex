\section{Présentation}

\subsection{Un peu d'histoire} % (fold)
    
    \LaTeX{} est né dans les années 1980.
    C'est une sorte de surcouche au système de composition de document \TeX{} écrite par Leslie Lamport.
    Son nom complet serait {\bfseries Lamport \TeX{}} mais il est souvent abrégé \LaTeX{}\footnote{Prononcez {\itshape "tek"} et {\itshape "latek"}}.
    
    Sa puissance est de pouvoir mettre en forme de manière propre nombre de documents allant du livre à l'article en passant par les rapports, les mémoires, les {\itshape slideshows} et autres lettres...

    Même si ce système peut sembler complexe et obscurs au débutant il reste extrèmement utilisé dans les milieux scientifiques.

    La personnalisation de \LaTeX{} est... complexe. Vous devrez souvent réécrire des commandes entières pour aboutir à une mise en page précise.
    Néammoins, cette personnalisation à l'extrème n'est pas nécessaire car \LaTeX{} offre déjà de nombreux modules et sa mise en page de base est relativement passe-partout.

% subsection Un peu d'histoire (end)

\subsection{Pourquoi utiliser \protect{\LaTeX{}} ?} % (fold)

    Outre le fait que ce soit sympa d'avoir enfin une vraie mise en page pour les documents scientifiques en général, une maîtrise de \LaTeX{} est parfois demandée en entreprise.

    \LaTeX{} {\bfseries n'est pas} un traitement de texte mais un {\itshape processeur de documents}.
    En d'autres termes, vous lui donnez du texte, il vous sort un document.
    Ce système vous permet de vous focaliser sur le contenu sans vous préocuper de la forme (ou qu'un peu).

    Même si au début \LaTeX vous semble une perte de temps, il faut bien comprendre qu'ensuite (notament dans le cas de documents importants), cela devient un outil surpuissant !

    Enfin, pour la petite histoire, sachez que \LaTeX{} et \TeX{} avant lui sont de vrais langages de programmation et que l'on trouve de vrais programmes écrits en \TeX{}\footnote{Regardez du côte de \url{http://mathim.com/} par exemple}.
    Même si de tels projets sont suicidaires, il est bon de savoir que {\bfseries c'est possible}.

% subsection Pourquoi utiliser \LaTeX{} ? (end)

\subsection{Quels formats ?} % (fold)
\label{sub:Quels formats ?}

Les docs \LaTeX{} se tapent dans le format de fichier {\verb .tex }.

Les fichiers de sortie seront soit des {\verb .ps } (format postscript), soit des {\verb .pdf } (format PDF).

Les images à inclure seront principalement des images en {\verb .png } (bitmap) ou {\verb .eps } ({\itshape encapsulated postscript}).
Certains programmes (GNUPlot par exemple) proposent aussi une exportation en {\verb .tex } pour les inclure plus facilement.

Ce dernier point nous amène à une autre information : il est possible de créer des images en \LaTeX{} au moyen de {\itshape packages} particuliers.

% subsection Quels formats ? (end)
