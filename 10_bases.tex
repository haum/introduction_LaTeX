\section{Les bases} % (fold)
\label{sec:Les bases}

Avant de vous lancer dans l'écriture de votre roman, on va peut être voir un peu les bases non ?

\subsection{Premier document} % (fold)
\label{sub:Premier document}

Comme premier document, on va utiliser un extrait de Wikipedia (parce que {\itshape Hello World} en \LaTeX, c'est inutile).
On va donc créer un document reprenant la section {\bfseries Synopsis} de la page sur \underline{Le Guide du Voyageur Galactique}\footnote{La page est consultable à l'adresse suivante : \url{http://fr.wikipedia.org/wiki/Le_Guide_du_voyageur_galactique}}.

Voilà le code, l'explication vient ensuite (et dans les commentaires) :

\begin{verbatim}
\documentclass[11pt, a4paper]{article}
% Modules à utiliser
\usepackage{url}
\usepackage[french]{babel}
\usepackage[T1]{fontenc}
\usepackage[utf8]{inputenc}

% Titre
\title{H2G2 (sur Wikipedia)}
\author{Douglas Adams}
\date{1982}

\begin{document}
    \maketitle % on affiche le titre

Arthur Dent, citoyen anglais moyen, assiste à la destruction de la Terre par
les Vogons afin de créer une voie express hyperspatiale, et est sauvé par son
ami Ford Prefect, qui est en fait originaire de Bételgeuse, et qui l'emmène
avec lui en astrostop à travers la Galaxie.
Les deux amis sont finalement récupérés par le vaisseau Cœur en Or, volé peu
de temps auparavant par Zaphod Beeblebrox, président de la galaxie et son
assistante Trillian.

Pour se guider, ils ont avec eux un exemplaire du Guide du voyageur galactique,
sorte de guide du routard pour la galaxie, célèbre pour sa phrase de couverture :
Don't Panic (ou Pas de panique dans la traduction française).

On apprend notamment dans le roman la réponse à la Grande Question sur la Vie,
l'Univers et le Reste.
S'amusant de ce que l'existence a finalement d'improbable en soi, l'auteur
développe ainsi le concept : la probabilité et son infinité de possibles donnent
lieu a un scénario débridé.
Mais pas de panique, le Guide est là comme point de repère.
L’œuvre est traitée avec beaucoup d'originalité, d’intelligence, mais surtout d'humour.

\end{document}
\end{verbatim}

Expliquons maintenant le code pas à pas.

Notons, avant de commencer que les commentaires sont indiqués par le symbole {\verb % }.
En \LaTeX, les commentaire sont forcément sur une seule ligne : tout ce qui suit ledit symbole est commentaire.

Tout d'abord, il faut repérer 3 instructions qui défissent 2 zones (zones que nous détaillerons ensuite) :

\begin{verbatim}
\documentclass[11pt, a4paper]{article}

% ... Préambule

\begin{document}

% .... corps du document

\end{document}
\end{verbatim}

Le {\bfseries préambule} contient une série de commandes permettant à \LaTeX{} de mettre le document en forme correctement.
C'est là que l'on spécifie les {\bfseries modules} à inclure, les commandes à (re)définir, etc...

Le {\bfseries corps du document} est entouré par les directives {\verb \begin{document} } et  {\verb \end{document} }.
C'est à cet endroit que l'on inscrira le contenu de notre document.

Si nous examinions d'un peu plus près la permière ligne : 
\begin{verbatim}
\documentclass[11pt, a4paper]{article}
\end{verbatim}

Le nom de la commande est explicite, cette ligne définie le type de document (sa {\itshape classe} que la commande prend en argument obligatoire.
{\verb documentclass } prend aussi d'autres options optionnelles (entre crochets). Ici :
\begin{description}
    \item[a4paper] On donne le type de papier cible (ici, une feuille A4 standard)
    \item[11pt] On spécifie ici la taille de police ({\itshape 11 points})
\end{description}

D'autres arguments existent, d'autre classes aussi :
\begin{description}
    \item[article] un article est une publication courte. En général, il ne comprend pas de page de titre et sa hiérarchie de titre commence à la {\verb section } (voir plus bas). 
    \item[report] il s'agit d'une publication un peu plus longue (voir carrément plus longue). Là, page de titre et tout le toutim, les titres commencent à la {\verb part }.
    \item[book] là, le nom vous fera comprendre qu'on attaque le sérieusement long. \LaTeX{} gère sans problème des {\itshape books } de plus de 1000 pages... On a la possibilité d'inclure {\verb part }, {\verb chapter } et autres...
    \item[letter] là encore, le nom de classe est explicite : il s'agit de lettre. certaines options permettent d'éditer des lettres présentés à la française. Notez que des commandes autorisent le remplissage des champs propres à la lettre.
    \item[beamer] la classe des fameux {\it slideshows}. On y reviendra.
    \item[] etc...
\end{description}

\subsubsection{Préambule} % (fold)
\label{ssub:Préambule}

    Comme dit plus haut, le préambule, c'est la {\itshape configuration } propre à ce document.

    Il contient plusieurs types d'instructions, voici les trois principaux :
    \begin{itemize}
        \item Les modules à inclure
        \item Les options de contenu (titre, date, auteur, etc...)
        \item Les (re)définitions de commandes
    \end{itemize}

\begin{verbatim}

%%% Modules à utiliser %%%
    % indique que le doc est en français 
    % (prise en compte de conventions typographiques)
\usepackage[french]{babel}
    % indique que l'on utilise des fontes (polices) comptatibles T1
\usepackage[T1]{fontenc}
    % indique le charset d'entrée (i.e. l'encodage), ici : unicode
\usepackage[utf8]{inputenc}
    % nous permet d'utiliser la commande \url{}
\usepackage{url}

%%%  Titre %%%
\title{H2G2 (sur Wikipedia)} % titre
\author{Douglas Adams}       % auteur
\date{1982}                  % date
\end{verbatim}

\paragraph{Les modules} % (fold)
\label{par:Les modules}

Les modules, ou {\it packages } dans la langue de Shakespeare, sont des ensemble de commandes permettant d'étendre le jeu de base.
Certains définissent des options pour le document (son encodage, les polices utilisées, etc...).

La commande pour l'inclusion de modules est de la forme :

\begin{verbatim}
    \usepackage[arguments optionnels]{nom du paquet}
\end{verbatim}

% paragraph Les modules (end)

\paragraph{Les commandes de contenu} % (fold)
\label{par:Les commandes de contenu}

Pour l'instant, nous n'avons vu que le titre. Sachez juste que, après le {\verb \begin{document} }, le titre est inlus grace à la commande : 
\begin{verbatim}
    \maketitle
\end{verbatim}
% paragraph Les commandes de contenu (end)

\paragraph{Les (re)définitions de commandes} % (fold)
\label{par:Les (re)définitions de commandes}

Ces commandes permettent de changer le comportement de certaines commandes voire d'en ajouter.
Ces instructions sont de la forme :

\begin{verbatim}
    % définition
    \newcommand{nouvelle commande}{effet}

    % redéfinition
    \renewcommand{commandeà modifier}{nouvel effet}
\end{verbatim}

Il existe des formes plus poussées, mais nous y reviendrons plus tard.
% paragraph Les (re)définitions de commandes (end)

% subsubsection Préambule (end)

\subsubsection{Corps du document} % (fold)
\label{ssub:Corps du document}

Dans un document \LaTeX, il n'y a aucun marquage particulier autour des paragraphes.
Tapez tout simplement le texte à la suite (pas forcément sur une même ligne, mais sans ligne vide), lorsque vous voudrez rompre le paragraphe courant, passez une ligne.

Il y a par contre un moyen de forcer le retour à la ligne {\bf sans changement de paragraphe} : ajouter 2 backslashes en à l'endroit de la césure. Exemple :

\begin{verbatim}
    Ceci est une ligne\\
    que je souhaite couper sans changer de paragraphe.

    Voici un nouveau paragraphe.
\end{verbatim}

Sachez aussi que \LaTeX{} se charge de la césure des mots automatiquement (et rajoute le trait d'union requis).


% subsubsection Corps du document (end)

% subsection Premier document (end)

\subsection{Ajouter un titre} % (fold)
\label{sub:Ajouter un titre}

Bon, là, j'ai un peu spoilé...

L'ajout d'un titre se fait via l'utilisation des commandes suivantes :

\begin{verbatim}
    \title{}
    \author{}
    \date{}
    % ...
    \maketitle
\end{verbatim}

Bien entendu, vous mettez ce que vous voulez ! Rien ne vous force à réellement mettre un nom d'auteur dans le champs {\verb author }.

Notez par contre que si la commande {\verb date } n'est pas appellée, \LaTeX rempli le champs en question avec la date courante. Si vous ne voulez pas de date, il suffit d'appeller la commande sans rien donner en argument :

\begin{verbatim}
    \date{}
\end{verbatim}

% subsection Ajouter un titre (end)

\subsection{Titres} % (fold)
\label{sub:Titres}

Bien entendu, certaines commandes permettent d'ajouter des titres. Sachez qu'ils sont {\bf automatiquement} numérotés, cette numérotation est d'ailleurs supprimable en suffixant le type de titre avec un astérisque ({\verb * }).

La hierarchie des titres les plus utilisés est la suivante :

\begin{enumerate}
    \item part
    \item chapter
    \item section
    \item subsection
    \item subsubsection
    \item paragraph
\end{enumerate}

Sachez aussi qu'il est possible de redéfinir la numérotation des titres avec les commandes

% subsection Titres (end)

\subsection{Mise en forme particulière} % (fold)
\label{sub:Mise en forme particulière}

Bon... on va pas commencer à utiliser {\bf sérieusement} un composeur de documents qui ne permet pas de mettre le texte en gras, en italique, de le souligner, etc...
De même, c'est plutôt utile de pouvoir l'ajuster à droite, à gauche ou au centre...

Voyons tout ça un peu plus en détail.

\subsubsection{Agissons sur la fonte} % (fold)
\label{ssub:Agissons sur la fonte}

Différente variantes de fonte existent, nous n'allons pas les détailler.

En effet, cette page recense largement assez de variantes pour que je ne m'embète pas à les recopier et à refaire le tableau :

\begin{center}
    \url{http://www.siteduzero.com/tutoriel-3-267238-les-polices.html}
\end{center}

Cette page met aussi en avant des fonctionnalités de changement de police et de couleur.
Elle est mieux faite que ce que j'aurais pu faire moi même et ne contient aucune erreur (du moins, j'en ai pas vu).

% subsubsection Agissons sur la fonte (end)

\subsubsection{Alignements} % (fold)
\label{ssub:Alignements}

Il est bon de savoir que \LaTeX{} justifie le texte par défaut. Il existe 3 {\bf environnements} pour justifier le texte :
\begin{itemize}
    \item {\verb flushright }
    \item {\verb center }
    \item {\verb flushleft }
\end{itemize}

Vous aurez, je pense compris que le premier aligne à droite, le second au centre, le troisème à droite.

Ces environnements s'utilisent comme tous les autres (que nous verons juste après) :
\begin{verbatim}
    \begin{environnement}
        texte dans l'environnement
    \end{environnement}
\end{verbatim}

Si je fais :
\begin{verbatim}
    \begin{center}
        She sells sea shells on the seashore.
    \end{center}
\end{verbatim}

Ça donne :
    \begin{center}
        She sells sea shells on the seashore.
    \end{center}

% subsubsection Alignements (end)

\subsection{Environnements} % (fold)
\label{sub:Environnements}

Voilà quelques autres environnements utiles.

Comme précédement, ces environnements s'appellent via :

\begin{verbatim}
    \begin{environnement}
        texte dans l'environnement
    \end{environnement}
\end{verbatim}

\subsubsection{Verse} % (fold)
\label{ssub:Verse}

L'environnement {\verb verse } permet la mise en page de poèmes :

\begin{verse}
    Horloge ! dieu sinistre, effrayant, impassible,\\
    Dont le doigt nous menace et nous dit: « Souviens-toi !\\
    Les vibrantes Douleurs dans ton cœur plein d’effroi\\
    Se planteront bientôt comme dans une cible ;\\
    \bigskip
    Le Plaisir vaporeux fuira vers l’horizon\\
    Ainsi qu’une sylphide au fond de la coulisse ;\\
    Chaque instant te dévore un morceau du délice\\
    À chaque homme accordé pour toute sa saison.\\

    \bigskip
    Trois mille six cents fois par heure, la Seconde\\
    Chuchote : Souviens-toi ! — Rapide, avec sa voix\\
    D’insecte, Maintenant dit : Je suis Autrefois,\\
    Et j’ai pompé ta vie avec ma trompe immonde !\\

    \bigskip
    Remember ! Souviens-toi, prodigue ! Esto memor !\\
    (Mon gosier de métal parle toutes les langues.)\\
    Les minutes, mortel folâtre, sont des gangues\\
    Qu’il ne faut pas lâcher sans en extraire l’or !\\

    \bigskip
    Souviens-toi que le Temps est un joueur avide\\
    Qui gagne sans tricher, à tout coup ! c’est la loi.\\
    Le jour décroît ; la nuit augmente; souviens-toi !\\
    Le gouffre a toujours soif ; la clepsydre se vide.\\

    \bigskip
    Tantôt sonnera l’heure où le divin Hasard,\\
    Où l’auguste Vertu, ton épouse encor vierge,\\
    Où le Repentir même (oh ! la dernière auberge !),\\
    Où tout te dira : Meurs, vieux lâche ! il est trop tard ! »
\end{verse}

(pour les incultes, ce poème de Baudelaire est paru dans \underline{Les Fleurs du Mal}, il s'intitule {\it L'Horloge})

% subsubsection Verse (end)

\subsubsection{Quotation/Quote} % (fold)
\label{ssub:Quotation/Quote}

Cet environnement met en forme les citations :

\begin{verbatim}
    \begin{quotation}
        Ils ne savaient pas que c'était impossible alors ils l'ont fait.
    \end{quotation}
    \begin{quote}
        Ils ne savaient pas que c'était impossible alors ils l'ont fait.
    \end{quote}
\end{verbatim}

\begin{quotation}
    Ils ne savaient pas que c'était impossible alors ils l'ont fait.
\end{quotation}
\begin{quote}
    Ils ne savaient pas que c'était impossible alors ils l'ont fait.
\end{quote}

Le deuxième ({\verb quote }) s'adapte mieux aux citations courtes alors que l'autre ({\verb quotation }) convient aux paragraphes complets.

% subsubsection Quotation/Quote (end)

\subsubsection{Verbatim} % (fold)
\label{ssub:Verbatim}

Un environnement dont j'abuse dans ce document est l'environnement {\verb verbatim } et son pendant {\it inline } {\verb \verb }.

Ces environnements permettent la mise en forme de code (même \LaTeX, ce document en est la preuve).

Voilà un exemple :

\begin{verbatim}
    L'environnement {\verb verbatim } permet la
    présentation et l'inclusion de code source.
\end{verbatim}

    L'environnement {\verb verbatim } permet la présentation et l'inclusion de code source.

{\verb verbatim } s'utilise comme n'importe quel autre environnement \LaTeX.


% subsubsection Verbatim (end)

% subsection Environnements (end)

\subsubsection{Listes} % (fold)
\label{ssub:Listes}

Il est bien évidement possible de créer des listes.

Avant tout, il est bon de savoir que les listes peuvent être de trois types :

\begin{itemize}
    \item Liste non-ordonnée (comme celle-ci)
    \item Liste ordonnée
    \item Liste associative
\end{itemize}

\paragraph{Liste non-ordonnée} % (fold)
\label{par:Liste non-ordonnée}

c'est la plus simple des listes.
L'environnement utilisé est {\verb itemize }.
chaque élément doit être précédé de {\verb \item }, comme dans l'exemple ci dessous : 

\begin{itemize}
    \item oui
    \item non
    \item avec les palmes
    \item sans les palmes
\end{itemize}

\begin{verbatim}
    \begin{itemize}
        \item oui
        \item non
        \item avec les palmes
        \item sans les palmes
    \end{itemize}
\end{verbatim}

% paragraph Liste non-ordonnée (end)

\paragraph{Liste ordonnée} % (fold)
\label{par:Liste ordonnée}

Elle est très proche de la première.
La différence réside dans le fait que les éléments soient cette fois numérotés.

L'environnement utilisé est {\verb enumerate }.
chaque élément doit être précédé de {\verb \item }, comme dans l'exemple ci dessous : 

\begin{enumerate}
    \item oui
    \item non
    \item avec les palmes
    \item sans les palmes
\end{enumerate}

\begin{verbatim}
    \begin{enumerate}
        \item oui
        \item non
        \item avec les palmes
        \item sans les palmes
    \end{enumerate}
\end{verbatim}

% paragraph Liste ordonnée (end)

\paragraph{Liste associative} % (fold)
\label{par:Liste associative}

Ces listes présentent des couples clé-valeur.

L'environnement utilisé est {\verb description }.
Chaque couple est formaté comme ceci :

\begin{verbatim}
    \item[clé] valeur
\end{verbatim}

Comme d'habitude, voici un petit exemple :

\begin{description}
    \item[Linux] Yeah !
    \item[BSD] Pas mal non plus o/
    \item[MacOS] Un truc avec des vrais morceaux d'iPapy dedans
    \item[Windows] Argh.... (sic)
\end{description}

\begin{verbatim}
    \begin{description}
        \item[Linux] Yeah !
        \item[BSD] Pas mal non plus o/
        \item[MacOS] Un truc avec des vrais morceaux d'iPapy dedans
        \item[Windows] Argh.... (sic)
    \end{description}
\end{verbatim}
% paragraph Liste associative (end)

% subsubsection Listes (end)

% subsection Mise en forme particulière (end)

On a plus ou moins fait le tour des bases de \LaTeX.
Ce que vous savez vous permettra de faire pas mal de choses, et pour le reste, il y a la fin de ce document...

% section Les bases (end)
